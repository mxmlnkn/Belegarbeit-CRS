% I started a table of these symbols in order to get some system in the confusing notations of the lecture

\newglossaryentry{i}{
    name=\ensuremath{i},
    description={Index der einen Stoff bezeichnet.},
    sort=symboli
}
\newglossaryentry{j}{
    name=\ensuremath{j},
    description={Index der ein Element bezeichnet.},
    sort=symbolj
}
\newglossaryentry{k}{
    name=\ensuremath{k},
    description={Index der eine Reaktion bezeichnet.},
    sort=symbolk
}
\newglossaryentry{kstrain}{
    name=\ensuremath{k},
    description={Streckungsrate. Beachte, dass \gls{k} die Reaktionsnummer bezeichnet, falls es als Index benutzt wird.},
    sort=symbolkstrain
}

\newglossaryentry{x}{
    name=\ensuremath{x},
    description={eindimensionale Raumkoordinate},
    sort=symbolx
}

\newglossaryentry{N_s}{
    name=\ensuremath{N_s},
    description={Anzahl Spezies/Stoffe},
    sort=symbolN_s
}
\newglossaryentry{n_i}{
    name=\ensuremath{n_i},
    description={Stoffmenge des Stoffes $i$ in $\SI{}{\mole}$. $i$ z.B. $\mathrm{O}_2$ oder $\mathrm{CH}_4$},
    sort=symboln_i
}
\newglossaryentry{n}{
    name=\ensuremath{n},
    description={Gesamtstoffmenge. $n:=\sum\limits_i^{N_s} n_i$},
    sort=symboln
}
\newglossaryentry{X_i}{
    name=\ensuremath{X_i},
    description={Molenbruch. $X_i:=\frac{n_i}{n}$},
    sort=symbolX_i
}
\newglossaryentry{m_i}{
    name=\ensuremath{m_i},
    description={Masse des Stoffes $i$},
    sort=symbolm_i
}
\newglossaryentry{m}{
    name=\ensuremath{m},
    description={Gesamtmasse. $m := \sum\limits_{i\in\text{Stoffe}} \gls{m_i} = \sum\limits_{j\in\text{Elemente}} \gls{m_j}$},
    sort=symbolm
}
\newglossaryentry{Y_i}{
    name=\ensuremath{Y_i},
    description={Massenbruch des Stoffes $i$ wie er real vorliegt $Y_i:=\frac{ \gls{m_i} }{ \gls{m} }$, vgl.~\gls{Y_ia}},
    sort=symbolY_i
}
\newglossaryentry{Y_ia}{
    name=\ensuremath{Y_{i,\text{a}}},
    description={Massenbruch des Stoffes $i$ wie er bei einer rein hydrodynamischen Simulation ohne jegliche Stoffumwandlung vorliegen würde. Das $\text{a}$ steht hier für Ausgangszustand (vor der Einführung von chemischen Reaktionen)},
    sort=symbolY_ia
}
\newglossaryentry{m_j}{
    name=\ensuremath{m_j},
    description={Masse des Elements $j$, z.B. C,H,O,$\ldots$},
    sort=symbolm_j
}
\newglossaryentry{Z_j}{
    name=\ensuremath{Z_j},
    description={Massenbruch der in den Stoffen enthaltenen Elemente $j$. $Z_j :=\frac{ \gls{m_j} }{ \gls{m} }$. Es wird nicht mehr \gls{Y} für diesen Massenbruch benutzt, da $Y_\mathrm{C}$ für reinen Kohlenstoff, also Ruß vorgesehen ist, während $Z_\mathrm{C}$ den Kohlenstoff in allen vorkommenden Stoffen wie $\mathrm{CO}_2,\mathrm{CH}_4,\ldots$ umfasst. Nicht zu verwechseln mit \gls{Z}!},
    sort=symbolZ_j
}
\newglossaryentry{M_i}{
    name=\ensuremath{M_i},
    description={Molare Masse des Stoffes $i$},
    sort=symbolM_i
}
\newglossaryentry{M_j}{
    name=\ensuremath{M_j},
    description={Molare Masse des Elementes $j$},
    sort=symbolM_j
}
\newglossaryentry{M}{
    name=\ensuremath{M},
    description={Gesamtmolare Masse $M:=\frac{ \gls{m} }{ \gls{n} } = \sum\limits_{i} \frac{ \gls{M_i} \gls{n_i} }{ \sum\limits_{i} \gls{n_i}} = \sum\limits_{i} \gls{M_i} \gls{X_i} $},
    sort=symbolM
}
\newglossaryentry{c_i}{
    name=\ensuremath{c_i},
    description={Stoffmengenkonzentration des Stoffes $i$ in $\SI{}{\mole\per\cubic\meter}$. In der Literatur auch unschöner Weise als $[X_i]$ bezeichnet, obwohl eckige Klammern für das Einheitenzeichen gemäß  EN ISO 80000 vorgesehen sind.},
    sort=symbolc_i
}

% Starting now are rather dull symbols

\newglossaryentry{a_ij}{
    name=\ensuremath{a_{ij}},
    description={Stoffkonfiguration. Eine Konstante, die die Anzahl an Atomen des Elements $j$ im Stoff $i$ angibt.},
    sort=symbola_ij
}
\newglossaryentry{X}{
    name=\ensuremath{X},
    description={Molenbruch. Verhältnis Partialstoffmenge zu Gestamtstoffmenge \gls{n}},
    sort=symbolX_i
}
\newglossaryentry{Y}{
    name=\ensuremath{Y},
    description={Massenbruch. Verhältnis Partialmasse zu Gestamtstoffmenge \gls{m}},
    sort=symbolY
}
\newglossaryentry{T}{
    name=\ensuremath{T},
    description={Temperatur. In dieser Arbeit häufig ein stationäres räumliches Temperaturfeld},
    sort=symbolT
}
\newglossaryentry{T_max}{
    name=\ensuremath{T_{\mathrm{max}}},
    description={Maximaler Wert eines stationären räumlichen Temperaturfeldes},
    sort=symbolTmax
}
\newglossaryentry{chi}{
    name=\ensuremath{\chi},
    description={Skalare Dissipationsrate, siehe Kapitel~\ref{sct:chist}},
    sort=symbolchi
}
\newglossaryentry{chi_st}{
    name=\ensuremath{\chi_{\mathrm{st}}},
    description={Skalare Dissipationsrate am Ort stöchiometrischer Mischung, siehe Kapitel~\ref{sct:chist}},
    sort=symbolchi_st
}
\newglossaryentry{Z}{
    name=\ensuremath{Z},
    description={Mischungsbruch, siehe Kapitel~\ref{sct:mischungsbruch}},
    sort=symbolZ
}
\newglossaryentry{Z_Bilger}{
    name=\ensuremath{Z_{\mathrm{Bilger}}},
    description={Bilger-Mischungsbruch, siehe Kapitel~\ref{sct:ZBilger}. Nicht zu verwechseln mit \gls{Z_j}!},
    sort=symbolZ_Bilger
}
\newglossaryentry{Z_ULF}{
    name=\ensuremath{Z_{\mathrm{ULF}}},
    description={Der Mischungsbruch wie er von \gls{ULF} berechnet uns ausgeben wird. Dieser sollte für $\mathrm{Le}=1$ mit \gls{Z_Bilger} übereinstimmen.},
    sort=symbolZ_Bilger
}
\newglossaryentry{PV}{
    name=\ensuremath{\mathrm{PV}},
    description={Fortschrittsvariable, siehe Kapitel~\ref{sct:PV}},
    sort=symbolPV
}
\newglossaryentry{nu_ik}{
    name=\ensuremath{\nu_{ik}},
    description={Stöchiometriefaktor (auch: stöchiometrische Zahl, stöchiometrischer Koeffizient, Stöchiometriezahl) des Stoffes $i$ bezüglich der Reaktion $k$, siehe Kapitel~\ref{sct:chemreact}. Nicht zu verwechseln mit \gls{nu}},
    sort=symbolnu_ik
}
\newglossaryentry{o_min}{
    name=\ensuremath{o_\text{min}},
    description={Massenbezogenen Mindestsauerstoffbedarf, siehe auch \gls{nu}},
    sort=symbolo_min
}
\newglossaryentry{nu}{
    name=\ensuremath{\nu},
    description={Identisch zu \gls{o_min}},
    sort=symbolnu
}

\newglossaryentry{x_st}{
    name=\ensuremath{x_\text{stöch}},
    description={Ort, an dem Brennstoff und Oxidator in stöchiometrischem Verhältnis vorliegen.},
    sort=symbolx_st
}
\newglossaryentry{v}{
    name=\ensuremath{v},
    description={Einströmgeschwindigkeit der beiden Düsen in der Simulation im Ortsraum. Beide Düsen haben die betragsmäßig gleiche aber entgegengesetzte Einströmgeschwindigkeit.},
    sort=symbolx_v
}
